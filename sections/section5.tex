\section{回溯法}
 
\subsection{回溯法的基本框架}

\subsubsection{递归回溯}

\textbf{思想说明:}
回溯法对解空间树进行\textbf{深度优先搜索}。
在一般情况下,回溯法采用\textbf{递归方式}实现。
搜索过程中,逐层构造解向量,当到达叶结点时输出一个解;
若当前部分解违反约束或不可能得到更优解,则进行剪枝。

\begin{lstlisting}[language=C,style=codeStyle]
void backtrack(int t)
{
    if (t > n) output(x);
    else
        for (int i = f(n,t); i <= g(n,t); i++) {
            x[t] = h(i);
            if (constraint(t) && bound(t))
                backtrack(t + 1);
        }
}
\end{lstlisting}

\textbf{复杂度分析:}
回溯法在最坏情况下需要遍历整个解空间树。
若每一层有 $b$ 个分支、深度为 $n$,
则最坏时间复杂度为$O(b^n).$
实际运行中,由于约束函数和限界函数的剪枝作用,
搜索规模通常会显著减小。

\subsubsection{迭代回溯}

\textbf{思想说明:}
递归回溯可以用\textbf{非递归方式}实现,
即采用树的非递归深度优先遍历方法。
迭代回溯通过显式维护层号 $t$,
模拟递归调用与回退过程。

\begin{lstlisting}[language=C,style=codeStyle]
void iterativeBacktrack()
{
    int t = 1;
    while (t > 0) {
        if (f(n,t) <= g(n,t))
            for (int i = f(n,t); i <= g(n,t); i++) {
                x[t] = h(i);
                if (constraint(t) && bound(t)) {
                    if (solution(t)) output(x);
                    else t++;
                }
            }
        else t--;
    }
}
\end{lstlisting}

\textbf{复杂度分析:}
迭代回溯与递归回溯本质上遍历的是\textbf{同一棵解空间树},
因此在最坏情况下时间复杂度相同,
仍为指数级 $O(b^n)$,
仅实现方式不同。

\subsubsection{子集树与排列树}

\textbf{子集树:}
子集树用于描述每个元素只有“取 / 不取”两种选择的问题,
例如 0-1 背包、装载问题、最大团问题等。
解空间是一棵\textbf{二叉树}。

\begin{lstlisting}[language=C,style=codeStyle]
void backtrack(int t)
{
    if (t > n) output(x);
    else
        for (int i = 0; i <= 1; i++) {
            x[t] = i;
            if (legal(t)) backtrack(t + 1);
        }
}
\end{lstlisting}

\textbf{复杂度分析:}
子集树共有 $2^n$ 个叶结点,
因此遍历子集树的时间复杂度为
$O(2^n).$

\textbf{排列树:}
排列树用于描述需要枚举元素排列顺序的问题,
例如旅行售货员问题、批处理作业调度问题等。
解空间是一棵\textbf{多叉树}。

\begin{lstlisting}[language=C,style=codeStyle]
void backtrack(int t)
{
    if (t > n) output(x);
    else
        for (int i = t; i <= n; i++) {
            swap(x[t], x[i]);
            if (legal(t)) backtrack(t + 1);
            swap(x[t], x[i]);
        }
}
\end{lstlisting}

\textbf{复杂度分析:}
排列树共有 $n!$ 个叶结点,
因此遍历排列树的时间复杂度为$O(n!).$

\subsection{装载问题(回溯法)}

\textbf{问题描述:}
有一批共 $n$ 个集装箱,需要装载到两艘轮船上,
两艘轮船的载重量分别为 $c_1$ 和 $c_2$。
第 $i$ 个集装箱的重量为 $w_i$,
且满足
$\sum_{i=1}^n w_i \le c_1 + c_2.$
装载问题要求判断是否存在一种合理的装载方案,
使得所有集装箱均能被装入这两艘轮船中。

\textbf{问题转化:}
若第一艘轮船尽可能多地装载集装箱,剩余集装箱自然装入第二艘轮船。因此,该问题可转化为:
从 $n$ 个集装箱中选择一个子集,
使其总重量不超过 $c_1$,
且尽可能接近 $c_1$。
这等价于一个特殊的 \textbf{0-1 背包问题},
可采用 \textbf{回溯法} 求解。

\begin{lstlisting}[language=C,style=codeStyle]
void backtrack(int i)
{
    if (i > n) {            // 到达叶结点
        bestw = cw;         // 更新最优解
        return;
    }
    r -= w[i];              // 更新剩余重量
    if (cw + w[i] <= c) {   // 搜索左子树(装第 i 个)
        x[i] = 1;
        cw += w[i];
        backtrack(i + 1);
        cw -= w[i];
    }
    if (cw + r > bestw) {   // 搜索右子树(不装第 i 个)
        x[i] = 0;
        backtrack(i + 1);
    }
    r += w[i];              // 回退到上一层
}
\end{lstlisting}

\textbf{复杂度分析:}
装载问题的解空间为一棵子集树,
共有 $2^n$ 个叶结点。
在最坏情况下,回溯算法需要遍历整个解空间,
其时间复杂度为$O(2^n).$
但通过可行性约束和上界函数进行剪枝,
实际搜索规模通常远小于 $2^n$。
在一般情况下,
若 $n$ 较大且要求多次求解,
可采用动态规划算法求解。

\subsection{批处理作业调度问题(回溯法)}

\textbf{问题描述:}
给定 $n$ 个作业的集合 $\{J_1,J_2,\dots,J_n\}$,
每个作业必须先在机器 $1$ 上加工,
然后在机器 $2$ 上加工。
作业 $J_i$ 在机器 $k$ 上的处理时间记为 $M[i][k]$。
对于一个确定的作业调度顺序,
设 $f_2$ 为所有作业在机器 $2$ 上完成加工的总时间,
则该调度的完成时间为 $f_2$。

\textbf{目标:}
确定一种作业加工顺序,
使得所有作业在机器 $2$ 上完成的总时间最小。

\begin{lstlisting}[language=C,style=codeStyle]
void Flowshop::Backtrack(int i)
{
    if (i > n) {
        for (int j = 1; j <= n; j++)
            bestx[j] = x[j];
        bestf = f;
    }
    else
        for (int j = i; j <= n; j++) {
            f1 += M[x[j]][1];
            f2[i] = (f2[i-1] > f1 ? f2[i-1] : f1) + M[x[j]][2];
            f += f2[i];

            if (f < bestf) {
                swap(x[i], x[j]);
                Backtrack(i + 1);
                swap(x[i], x[j]);
            }
            f1 -= M[x[j]][1];
            f -= f2[i];
        }
}
\end{lstlisting}

\textbf{复杂度分析:}
批处理作业调度问题的解空间是一棵排列树,
最多包含 $n!$ 个叶结点。
在最坏情况下,回溯算法需要遍历整个排列空间,
时间复杂度为$O(n!).$
通过限界函数进行剪枝,
可以在实际运行中大幅减少搜索结点数,
但该问题本质上仍是一个 \textbf{NP-难问题}。

\subsection{符号三角形问题(回溯法)}

\textbf{问题描述:}
符号三角形由符号 “$+$” 和 “$-$” 组成。
设第一行有 $n$ 个符号,其余各行由上一行相邻两个符号决定:
若两个符号相同,则下一行对应位置为 “$+$”;
若两个符号不同,则下一行对应位置为 “$-$”。
整个三角形共有$\frac{n(n+1)}{2}$个符号。
符号三角形问题要求:  
\textbf{对于给定的 $n$,计算有多少种不同的符号三角形,使得其中 “$+$” 和 “$-$” 的个数相同。}

\begin{lstlisting}[language=C,style=codeStyle]
void Triangle::Backtrack(int t)
{
    if ((count > half) || (t*(t+1)/2 - count > half))
        return;
    if (t > n)
        sum++;
    else
        for (int i = 0; i < 2; i++) {   // {0,1} -> {+, -}
            p[1][t] = i;
            count += i;
            for (int j = 2; j <= t; j++) {
                p[j][t-j+1] = p[j-1][t-j+1] + p[j-1][t-j+2];
                count += p[j][t-j+1];
            }
            Backtrack(t + 1);
            for (int j = 2; j <= t; j++)
                count -= p[j][t-j+1];
            count -= i;
        }
}
\end{lstlisting}

\textbf{复杂度分析:}该问题的解空间为一棵子集树,规模为 $2^n$;
每一次可行性判断与符号生成的代价为 $O(n)$;
因此在最坏情况下,算法时间复杂度为$O(n2^n).$
通过可行性约束进行剪枝,
可以显著减少实际搜索规模,
但该问题本质上仍属于指数级复杂度问题。

\subsection{n 后问题(回溯法)}

\textbf{问题描述:}
在一个 $n \times n$ 的棋盘上放置 $n$ 个皇后,
使得任意两个皇后之间都不能互相攻击。
按照国际象棋规则,
皇后可以攻击与其处在同一行、同一列或同一条对角线上的棋子。
$n$ 后问题要求计算在棋盘上放置 $n$ 个皇后的所有可行方案数。

\textbf{可行性检测函数:}
\begin{lstlisting}[language=C,style=codeStyle]
bool Queen::Place(int k)
{
    for (int j = 1; j < k; j++)
        if ((abs(x[j] - x[k]) == abs(j - k)) || (x[j] == x[k]))
            return false;
    return true;
}
\end{lstlisting}

\textbf{回溯算法代码:}

\begin{lstlisting}[language=C,style=codeStyle]
void Queen::Backtrack(int t)
{
    if (t > n)
        sum++;
    else
        for (int i = 1; i <= n; i++) {
            x[t] = i;
            if (Place(t))
                Backtrack(t + 1);
        }
}
\end{lstlisting}

\textbf{复杂度分析:}
$n$ 后问题的解空间是一棵排列树,
最多包含 $n!$ 个叶结点。
在最坏情况下,
回溯算法需要遍历整个排列空间,
其时间复杂度为
$O(n!).$
通过可行性检测函数进行剪枝,
可以显著减少实际搜索规模,
但该问题仍属于指数时间复杂度问题。


\subsection{0-1 背包问题(回溯法)}

\textbf{问题描述:}
给定 $n$ 个物品,
第 $i$ 个物品的重量为 $w_i$,价值为 $p_i$,
背包容量为 $c_1$。
每个物品只能选择“装入”或“不装入”一次。
目标是在不超过背包容量的前提下,
使装入物品的总价值最大。

\textbf{回溯算法代码:}
\begin{lstlisting}[language=C,style=codeStyle]
template<class Typew, class Typep>
void Knap<Typew, Typep>::Backtrack(int i)
{
    if (i > n) {
        bestp = cp;
        return;
    }
    if (cw + w[i] <= c) {
        cw += w[i];
        cp += p[i];
        Backtrack(i + 1);
        cw -= w[i];
        cp -= p[i];
    }
    if (Bound(i + 1) > bestp)
        Backtrack(i + 1);
}
\end{lstlisting}

\textbf{上界函数代码:}
\begin{lstlisting}[language=C,style=codeStyle]
template<class Typew, class Typep>
Typep Knap<Typew, Typep>::Bound(int i)
{
    Typew cleft = c - cw;
    Typep b = cp;
    while (i <= n && w[i] <= cleft) {
        cleft -= w[i];
        b += p[i];
        i++;
    }
    if (i <= n)
        b += p[i] / w[i] * cleft;
    return b;
}
\end{lstlisting}

\textbf{复杂度分析:}
0-1 背包问题的解空间为一棵子集树,
最坏情况下共有 $2^n$ 个叶结点。
回溯算法在最坏情况下需要遍历整个解空间,
时间复杂度为$O(2^n).$
通过上界函数进行剪枝,
可以显著减少实际搜索规模,
但该问题本质上仍属于指数时间复杂度问题。

\subsection{最大团问题(回溯法)}

\textbf{问题描述:}
给定无向图 $G=(V,E)$。
若 $U\subseteq V$,且对任意 $u,v\in U$,都有 $(u,v)\in E$,
则称 $U$ 为图 $G$ 的一个团。
若 $U$ 不包含于任何更大的团中,则称其为极大团。
图 $G$ 中顶点数最多的团称为 \textbf{最大团}。
最大团问题要求找出图 $G$ 的一个最大团。

\begin{lstlisting}[language=C,style=codeStyle]
void Clique::Backtrack(int i)
{
    if (i > n) {
        for (int j = 1; j <= n; j++)
            bestx[j] = x[j];
        return;
    }
    bool OK = true;
    for (int j = 1; j < i; j++)
        if (x[j] && !a[i][j]) {
            OK = false;
            break;
        }
    if (OK) {
        x[i] = 1;
        cn++;
        Backtrack(i + 1);
        cn--;
    }
    if (cn + n - i > bestn) {
        x[i] = 0;
        Backtrack(i + 1);
    }
}
\end{lstlisting}

\textbf{复杂度分析:}
最大团问题的解空间规模为 $2^n$。
在最坏情况下,回溯算法需要遍历整个子集树,
其时间复杂度为
$O(2^n).$
该问题是典型的 NP-难问题。

\subsection{图的 m 着色问题(回溯法)}

\textbf{问题描述:}
给定无向连通图 $G=(V,E)$ 和 $m$ 种不同的颜色,
要求用这 $m$ 种颜色对图中每个顶点着色,
使得任意一条边的两个端点颜色不同。
判断图是否存在一种合法的 $m$ 着色方案。

\textbf{回溯算法代码:}
\begin{lstlisting}[language=C,style=codeStyle]
void Color::Backtrack(int t)
{
    if (t > n)
        sum++;
    else
        for (int i = 1; i <= m; i++) {
            x[t] = i;
            if (OK(t))
                Backtrack(t + 1);
        }
}
\end{lstlisting}

\textbf{可行性检测函数:}
\begin{lstlisting}[language=C,style=codeStyle]
bool Color::OK(int k)
{
    for (int j = 1; j < k; j++)
        if (a[k][j] && x[j] == x[k])
            return false;
    return true;
}
\end{lstlisting}

\textbf{复杂度分析:}
图的 m 着色问题的解空间为 $m^n$。
在最坏情况下,
回溯算法需要遍历整个解空间,
时间复杂度为$O(m^n).$该问题是 NP-完全问题。

\subsection{旅行售货员问题(回溯法)}

\textbf{问题描述:}
设有 $n$ 个城市,
已知任意两个城市之间的距离。
旅行售货员从某一城市出发,
需要恰好访问每个城市一次,
并最终返回出发城市,
要求使得总旅行距离最短。

\begin{lstlisting}[language=C,style=codeStyle]
template<class Type>
void Traveling<Type>::Backtrack(int i)
{
    if (i == n) {
        if (a[x[n-1]][x[n]] != NoEdge &&
            a[x[n]][x[1]] != NoEdge &&
            (cc + a[x[n-1]][x[n]] + a[x[n]][x[1]] < bestc
             || bestc == NoEdge)) {
            bestc = cc + a[x[n-1]][x[n]] + a[x[n]][x[1]];
        }
    }
    else
        for (int j = i; j <= n; j++) {
            if (a[x[i-1]][x[j]] != NoEdge &&
                (cc + a[x[i-1]][x[j]] < bestc
                 || bestc == NoEdge)) {
                swap(x[i], x[j]);
                cc += a[x[i-1]][x[i]];
                Backtrack(i + 1);
                cc -= a[x[i-1]][x[i]];
                swap(x[i], x[j]);
            }
        }
}
\end{lstlisting}

\textbf{复杂度分析:}
旅行售货员问题的解空间规模为 $(n-1)!$。
在最坏情况下,
回溯算法需要遍历所有排列,
时间复杂度为$O(n!).$
该问题是经典的 NP-难问题。

\subsection{圆排列问题(回溯法)}

\textbf{问题描述:}
给定 $n$ 个大小不等的圆,
其半径分别为 $c_1,c_2,\dots,c_n$。
现要求将这 $n$ 个圆排列在一条水平直线上,
使所有圆都与水平线相切,
且整体排列所占的长度最小。
圆排列问题要求确定一种排列顺序,
使得最左端到最右端的距离最小。

\textbf{回溯算法代码:}
\begin{lstlisting}[language=C,style=codeStyle]
void Circle::Backtrack(int t)
{
    if (t > n)
        Compute();
    else
        for (int j = t; j <= n; j++) {
            swap(r[t], r[j]);
            if (Center(t) < best)
                Backtrack(t + 1);
            swap(r[t], r[j]);
        }
}
\end{lstlisting}

\textbf{圆心计算函数:}
\begin{lstlisting}[language=C,style=codeStyle]
float Circle::Center(int t)
{
    float temp = 0;
    for (int j = 1; j < t; j++) {
        float value = x[j] + 2.0 * sqrt(r[j] * r[t]);
        if (value > temp)
            temp = value;
    }
    x[t] = temp;
    float low = 0, high = 0;
    for (int i = 1; i <= t; i++) {
        if (x[i] - r[i] < low)
            low = x[i] - r[i];
        if (x[i] + r[i] > high)
            high = x[i] + r[i];
    }
    if (high - low < best)
        best = high - low;
    return x[t];
}
\end{lstlisting}

\textbf{复杂度分析:}
圆排列问题的解空间为一棵排列树,
最坏情况下需要枚举 $n!$ 种排列。
在每个结点需要 $O(n)$ 时间计算排列长度,
因此算法在最坏情况下的时间复杂度为
$O(n \cdot n!).$

\subsection{连续邮资问题(回溯法)}

\textbf{问题描述:}
假设国家发行了 $n$ 种不同面值的邮票,
并规定每封信封上最多允许贴 $m$ 张邮票。
连续邮资问题要求确定这 $n$ 种邮票的面值,
使得在最多贴 $m$ 张邮票的条件下,
从 $1$ 开始可以连续表示的邮资区间尽可能大。

\begin{lstlisting}[language=C,style=codeStyle]
void Stamp::Backtrack(int t)
{
    if (t > n) {
        if (r[n] > maxvalue) {
            maxvalue = r[n];
            for (int j = 1; j <= n; j++)
                bestx[j] = x[j];
        }
        return;
    }
    int *z = new int[maxx + 1];
    for (int i = 1; i <= maxx; i++)
        z[i] = y[i];

    for (int i = x[t-1] + 1; i <= r[t-1] + 1; i++) {
        x[t] = i;
        for (int k = 1; k <= maxx; k++)
            y[k] = z[k];
        Update(t);
        Backtrack(t + 1);
    }
    for (int k = 1; k <= maxx; k++)
        y[k] = z[k];
    delete[] z;
}
\end{lstlisting}

\textbf{复杂度分析:}
连续邮资问题的解空间规模随 $n$ 和 $m$ 急剧增长,
在最坏情况下,
回溯算法需要枚举大量组合,
时间复杂度为指数级。