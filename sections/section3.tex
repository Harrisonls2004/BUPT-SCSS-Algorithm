\section{动态规划}

\subsection{矩阵连乘问题}

\textbf{问题描述:}
给定一系列矩阵 $A_1, A_2, \dots, A_n$,
其中矩阵 $A_i$ 的规模为 $p_{i-1}\times p_i$。
由于矩阵乘法满足结合律,但不同的加括号方式会导致
\textbf{标量乘法次数不同},
矩阵连乘问题的目标是:\textbf{确定一种加括号方式,使得计算 $A_1A_2\cdots A_n$ 所需的标量乘法次数最少。}

状态转移方程为:$$
m[i,j] = 
\begin{cases} 
0, & i = j \\[8pt]
\displaystyle\min_{i \leq k < j} 
\bigl\{ m[i,k] + m[k+1,j] + p_{i-1}p_k p_j \bigr\}, & i < j 
\end{cases}
$$

\subsubsection{常规方法}
\begin{lstlisting}[language=C,style=codeStyle]
void MatrixChain(int *p, int n, int **m, int **s) 
// m[i][j]:最少乘法次数
// s[i][j]:A_i...A_j 的最优断开位置
void MatrixChain(int* p, int n, int** m, int** s) {
    for(int i = 1; i <= n; i++) m[i][i] = 0;
    for(int r = 2; r <= n; r++) {
        for(int i = 1; i <= n - r + 1; i++) {
            int j = i + r - 1;
            m[i][j] = m[i+1][j] + p[i-1]*p[i]*p[j];
            s[i][j] = i;
            for(int k = i + 1; k < j; k++) {
                int t = m[i][k] + m[k+1][j] + p[i-1]*p[k]*p[j];
                if(t < m[i][j]) {
                    m[i][j] = t;
                    s[i][j] = k;
                }
            }
        }
    }
}
\end{lstlisting}

\textbf{算法复杂度分析:}MatrixChain的主要计算量取决于算法中对$i$,$j$和$k$的3重循环。循环体内的计算量为$O(1)$,而3重循环的总次数为$O(n^{3})$。因此算法的计算时间上界为$O(n^{3})$。算法所占用的空间显然为$O(n^{2})$。

\subsubsection{递归方法}

\begin{lstlisting}[language=C,style=codeStyle]
int LookupChain(int i, int j)
{
    if (m[i][j] > 0) return m[i][j];      // 已算过:直接返回(记忆化)
    if (i == j) return 0;                 // 只有一个矩阵:代价为0
    int u = LookupChain(i, i) + LookupChain(i+1, j) + p[i-1] * p[i] * p[j]; 
    // 先假设 k = i
    s[i][j] = i;
    for (int k = i+1; k < j; k++) {       // 枚举所有划分点 k
        int t = LookupChain(i, k) + LookupChain(k+1, j) + p[i-1] * p[k] * p[j];
        if (t < u) {                      // 取最小
            u = t;
            s[i][j] = k;                  // 记录最优划分点
        }
    }
    m[i][j] = u;                          // 记忆化保存结果
    return u;
}
\end{lstlisting}

\textbf{递归复杂度:}
子问题总数为 $\frac{n(n+1)}{2}=O(n^2).$
每个子问题在求最小值时要枚举 $k$,最多 $O(n)$ 次,
所以总时间复杂度为 $O(n^3),$
空间复杂度由 $m,s$ 两张表决定,为 $O(n^2).$

\subsection{最长公共子序列(LCS)}
\textbf{问题描述:}
给定两个序列 $X=\langle x_1,x_2,\dots,x_m\rangle$ 与
$Y=\langle y_1,y_2,\dots,y_n\rangle$,
求它们的最长公共子序列(LCS)的长度,并输出一个 LCS。

$$
c[i][j]=
\begin{cases}
0, & i=0, j=0 \\[6pt]
c[i-1][j-1]+1, & i,j>0;\ x_i = y_j \\[6pt]
\max\{c[i][j-1],\ c[i-1][j]\}, & i,j>0;\ x_i \neq y_j
\end{cases}
$$

\begin{lstlisting}[language=C,style=codeStyle]
void LCSLength(int m, int n, char *x, char *y, int **c, int **b)
{
    int i, j;
    // 初始化边界:任意序列与空序列的LCS长度为0
    for (i = 0; i <= m; i++) c[i][0] = 0;
    for (j = 0; j <= n; j++) c[0][j] = 0;
    // 动态规划填表:从小规模子问题推到大规模子问题
    for (i = 1; i <= m; i++) {
        for (j = 1; j <= n; j++) {
            if (x[i] == y[j]) {              // xi == yj
                c[i][j] = c[i-1][j-1] + 1;
                b[i][j] = 1;                 // 记录来自左上
            } else if (c[i-1][j] >= c[i][j-1]) {
                c[i][j] = c[i-1][j];
                b[i][j] = 2;                 // 记录来自上
            } else {
                c[i][j] = c[i][j-1];
                b[i][j] = 3;                 // 记录来自左
            }
        }
    }
}
void LCS(int i, int j, char *x, int **b)
{
    if (i == 0 || j == 0) return;     // 回溯到边界,结束
    if (b[i][j] == 1) {               // 来自左上:说明匹配到一个字符
        LCS(i-1, j-1, x, b);
        printf("%c", x[i]);           // 输出该字符(属于LCS)
    } else if (b[i][j] == 2) {        // 来自上:丢弃 xi
        LCS(i-1, j, x, b);
    } else {                          // 来自左:丢弃 yj
        LCS(i, j-1, x, b);
    }
}
\end{lstlisting}

\textbf{复杂度:}
表格大小为 $m\times n$,每个格子 $O(1)$ 计算,
所以时间复杂度为 $O(mn),$
空间复杂度为 $O(mn).$
回溯输出最多走 $m+n$ 步,为 $O(m+n).$

\subsection{最优三角形划分(Min-Weight Triangulation)}

\textbf{问题描述:}
给定一个凸多边形,顶点按顺序编号为
$v_0, v_1, v_2, \dots, v_n,$
其中共有 $n+1$ 个顶点。
将多边形用不相交的对角线划分成若干三角形(三角剖分)。
设三角形 $(v_i,v_k,v_j)$ 的代价为
$w(i,k,j).$
目标是找到一种三角剖分,使总代价最小。

对所有 $i\le k<j$ 取最小即可。$$
t[i][j]=
\begin{cases}
0, & i=j \\[8pt]
\displaystyle\min_{i\leq k<j}\{ t[i][k]+t[k+1][j]+w(v_{i-1}v_{k}v_{j})\}, & i<j
\end{cases}
$$

\begin{lstlisting}[language=C,style=codeStyle]
void MinWeightTriangulation(int n, Type **t, int **s)
{
    for (int i = 1; i <= n; i++) t[i][i] = 0;
    for (int r = 2; r <= n; r++) { // 子问题长度 r: j = i + r - 1
        for (int i = 1; i <= n - r + 1; i++) {
            int j = i + r - 1;
            // 先令 k = i 作为初值(即选三角形 (i-1, i, j))
            t[i][j] = t[i+1][j] + w(i-1, i, j);
            s[i][j] = i;
            // 枚举 i+1 ... j-1
            for (int k = i+1; k < j; k++) {
                int u = t[i][k] + t[k+1][j] + w(i-1, k, j);
                if (u < t[i][j]) {
                    t[i][j] = u;
                    s[i][j] = k;
                }
            }
        }
    }
}
\end{lstlisting}

\textbf{复杂度分析:}
状态数为 $O(n^2)$(所有 $1\le i\le j\le n$),
每个状态需要枚举 $k$,最多 $O(n)$ 次,
因此时间复杂度为
$O(n^3).$
数组 $t,s$ 的规模为 $O(n^2)$,空间复杂度为
$O(n^2).$

\subsection{多边形游戏(Polygon Game)}

\textbf{问题描述:}
多边形游戏给定一个含 $n$ 个顶点的环状表达式(可看作一个多边形),
每条边上有一个运算符(如 $+$ 或 $\times$),每个顶点上有一个数。
通过选择删边/断开位置并确定计算顺序(等价于给表达式加括号),
最终会得到一个结果值。目标通常是求\textbf{最大结果}(或同时求最大/最小以便处理乘法含负数的情况)。

\[
\min f(i,j,s) = 
\begin{cases}
a + c, & \text{op}[i+s] = '+' \\[6pt]
\min\{ac, ad, bc, bd\}, & \text{op}[i+s] = '*'
\end{cases}
\]

\[
\max f(i,j,s) = 
\begin{cases}
b + d, & \text{op}[i+s] = '+' \\[6pt]
\max\{ac, ad, bc, bd\}, & \text{op}[i+s] = '*'
\end{cases}
\]

由于最优断开位置 $s$ 有 $1 \leq s \leq j-1$ 的 $j-1$ 种情况,由此可知

\[
m[i,j,0] = \min_{1 \leq s \leq j-1} \{\min f(i,j,s)\}, \quad 1 \leq i,j \leq n
\]

\[
m[i,j,1] = \max_{1 \leq s \leq j-1} \{\max f(i,j,s)\}, \quad 1 \leq i,j \leq n
\]

\textbf{初始边界值:}
$
m[i,1,0] = v[i], \quad m[i,1,1] = v[i]
$

\begin{lstlisting}[language=C,style=codeStyle]
void PolyMax(int n)
{
    int minf, maxf;
    for (int j = 2; j <= n; j++) {          // 子链长度 j
        for (int i = 1; i <= n; i++) {      // 子链起点 i(环状)
            for (int s = 1; s < j; s++) {   // 断开位置 s(划分成两段)
                MinMax(n, i, s, j, minf, maxf);
                if (m[i][j][0] > minf)
                    m[i][j][0] = minf;      // 更新最小值
                if (m[i][j][1] < maxf)
                    m[i][j][1] = maxf;      // 更新最大值
            }
        }
    }
    int temp = m[1][n][1];
    for (int i = 2; i <= n; i++) {          // 枚举起点,取全局最大
        if (temp < m[i][n][1])
            temp = m[i][n][1];
    }
    return temp;
}
\end{lstlisting}

\textbf{复杂度分析:}
三重循环规模约为:
$j:O(n),\ i:O(n),\ s:O(n),$
每次 \texttt{MinMax} 计算为常数时间,
因此时间复杂度为$O(n^3),$
状态数组 $m[i][j][0/1]$ 的规模为 $O(n^2)$,
空间复杂度为$O(n^2).$

\subsection{图像压缩}

\textbf{问题描述:}
给定灰度像素序列 $\{p_1,p_2,\dots,p_n\}$,其中 $0\le p_i \le 255$。
将序列划分为若干个\textbf{连续段} $S_1,S_2,\dots,S_m$。
第 $i$ 段记为
$S_i=\{p_{t_{i-1}+1},\dots,p_{t_i}\},$
并用同一位宽存储该段内所有像素。
设该段所需位宽为
$b_i=\left\lceil \log_2\left(\max_{t_{i-1}+1\le k\le t_i}p_k + 1\right)\right\rceil,$
且由题意有 $b_i\le 8$,因此存储 $b_i$ 需要 3 位;
段长 $l_i=t_i-t_{i-1}$ 满足 $1\le l_i\le 255$,存储 $l_i$ 需要 8 位。
因此每段头部开销为
$3+8=11\ \text{位}.$
目标是:\textbf{选择一种分段方式,使总存储位数最小},且每段长度不超过 256。

$$
S[i] = \min_{1 \le j \le \min(i,256)}
\left\{ S[i-j] + j \cdot \max_{i-j+1 \le t \le i} b[t] \right\} + 1,
\qquad
b[t] = \text{length}(p_t).
$$

\begin{lstlisting}[language=C,style=codeStyle]
void Compress(int n, int p[], int S[], int L[], int b[])
{
    int Lmax = 256, header = 1;
    S[0] = 0;
    for (int i = 1; i <= n; i++) {           // 依次计算 S[1..n]
        b[i] = length(p[i]);                 // 第 i 个元素所需位数
        int bmax = b[i];
        S[i] = S[i-1] + bmax;                // 先假设最后一段只含 p[i]
        L[i] = i;
        for (int j = 2; j <= i && j <= Lmax; j++) {   // j 表示最后一段长度
            if (bmax < b[i-j+1]) bmax = b[i-j+1];      // 更新该段最大位宽
            if (S[i] > S[i-j] + j * bmax) {            // 选择更优的分段
                S[i] = S[i-j] + j * bmax;
                L[i] = j;                               // 记录最后一段长度
            }
        }
        S[i] += header;                                 // 加上该段头部开销
    }
}
\end{lstlisting}

\textbf{复杂度分析:}
对每个 $i$,内层最多枚举 $j=1.. \min(i,Lmax)$,
故时间复杂度为
$O(n\cdot Lmax).$
当 $Lmax=256$ 为常数时,可写为 $O(n).$
空间复杂度主要来自数组 $S,L,b$,为 $O(n).$

\subsection{电路布线问题(MNS)}

\textbf{问题描述:}
在电路布线问题中,给定若干条连线,
每条连线用一对端点 $(i,\pi(i))$ 表示,
其中 $i$ 表示左侧端点编号,$\pi(i)$ 表示其在右侧对应的端点编号。
若两条连线 $(i,\pi(i))$ 与 $(t,\pi(t))$ 满足
$i<t\ \text{但}\ \pi(i)>\pi(t),$
则称这两条连线\textbf{相交}。
问题的目标是:\textbf{从所有连线中选取一个最大的不相交子集},使得任意两条被选中的连线都不相交。

\textbf{边界条件:}
当 $i=1$ 时
\[
\text{Size}(1,j)=
\begin{cases}
0, & j<\pi(1),\\
1, & j\ge \pi(1).
\end{cases}
\]

\textbf{状态转移方程:}
当 $i>1$ 时,有如下递推关系:
\[
\text{Size}(i,j)=
\begin{cases}
\text{Size}(i-1,j), & j<\pi(i),\\[6pt]
\max\bigl\{\text{Size}(i-1,j),\ \text{Size}(i-1,\pi(i)-1)+1\bigr\}, & j\ge \pi(i).
\end{cases}
\]

\begin{lstlisting}[language=C,style=codeStyle]
void MNS(int C[], int n, int **size)
{
    // 初始化 i=1 的情况
    for (int j = 0; j < C[1]; j++) size[1][j] = 0;
    for (int j = C[1]; j <= n; j++) size[1][j] = 1;
    // 填表
    for (int i = 2; i <= n; i++) {
        for (int j = 0; j < C[i]; j++)
            size[i][j] = size[i-1][j];
        for (int j = C[i]; j <= n; j++)
            size[i][j] = max(size[i-1][j], size[i-1][C[i]-1] + 1);
    }
    // 最终结果
    size[n][n] = max(size[n-1][n],size[n-1][C[n]-1] + 1);
}
\end{lstlisting}

\textbf{最优解构造(回溯):}

\begin{lstlisting}[language=C,style=codeStyle]
void Traceback(int C[], int **size, int n, int Net[], int &m)
{
    int j = n;
    m = 0;
    for (int i = n; i > 1; i--) {
        if (size[i][j] != size[i-1][j]) {
            Net[m++] = i;          // 选择连线 i
            j = C[i] - 1;
        }
    }
    if (j >= C[1])
        Net[m++] = 1;
}
\end{lstlisting}

\textbf{复杂度分析:}
状态表规模为 $O(n^2)$,
每个状态计算时间为 $O(1)$,
因此时间复杂度为$O(n^2),$
空间复杂度为$O(n^2).$
回溯过程仅需一次线性扫描,时间复杂度为 $O(n)$。

\subsection{流水作业调度(两台机器 $M_1,M_2$)}

\textbf{问题描述:}
有 $n$ 个作业 $1,2,\dots,n$,需要在两台机器 $M_1,M_2$ 上依次加工。
每个作业 $i$ 必须先在 $M_1$ 上加工 $a_i$ 时间,再在 $M_2$ 上加工 $b_i$ 时间(不允许改变顺序)。
要求安排一个作业加工顺序,使得:
第一个作业在机器 $M_1$ 上开始加工,
最后一个作业在机器 $M_2$ 上完成加工的时间(即\textbf{完工时间})最小。

\textbf{状态转移方程:}
由流水作业调度问题的最优子结构性质可知,
\[
T(N,0) = \min_{1 \leq i \leq n} \left\{ a_{i} + T(N-\{i\}, b_{i}) \right\}
\]

\[
T(S,t) = \min_{i \in S} \left\{ a_{i} + T(S-\{i\}, b_{i} + \max\{t-a_{i}, 0\}) \right\}
\]

其中:
\begin{itemize}
  \item $a_j$:作业 $j$ 在 $M_1$ 的加工时间;
  \item $T(S-\{j\},b_j)$:剩余作业子集的最优等待时间(将 $b_j$ 作为状态参数);
  \item $\max(t-a_j,0)$:由于 $M_2$ 的空闲/等待产生的额外影响项。
\end{itemize}

\textbf{结论:}
流水作业调度问题在两台机器情形下具有最优子结构性质,
可用集合型动态规划描述,其最优完成时间为
$T(N,0).$

\subsection{0-1 背包问题(特殊的整数规划问题)}

\textbf{问题描述:}
给定 $n$ 种物品和一个背包。第 $i$ 个物品的重量为 $w_i$、价值为 $v_i$,
背包容量为 $C$。每个物品只能取 $0$ 或 $1$ 次(不可分割)。
目标是在不超过容量 $C$ 的前提下,使装入背包的物品总价值最大。

设决策变量
$$
x_i=
\begin{cases}
1, & \text{选取物品 } i,\\
0, & \text{否则},
\end{cases}
\qquad i=1,2,\dots,n.
$$

则模型为
$$
\max \sum_{i=1}^{n} v_i x_i
$$

满足约束
$$
\sum_{i=1}^{n} w_i x_i \le C,\qquad x_i\in\{0,1\},\ 1\le i\le n.
$$

\textbf{子问题与状态定义:}
定义子问题:当\textbf{可选物品为 $i,i+1,\dots,n$},
背包\textbf{容量为 $j$}($0\le j\le C$)时的最优值记为
$m(i,j).$
也就是说,$m(i,j)$ 表示在约束
$\sum_{k=i}^{n} w_k x_k \le j,\qquad x_k\in\{0,1\}\ (i\le k\le n)$下最大化
$\sum_{k=i}^{n} v_k x_k$所能得到的最优目标值。

\textbf{边界条件:}
当只剩最后一个物品 $n$ 可选时,
$$
m(n,j)=
\begin{cases}
v_n, & j\ge w_n,\\
0, & 0\le j < w_n.
\end{cases}
$$

\textbf{状态转移方程(最优子结构):}
对 $i<n$,考虑物品 $i$ 的“选/不选”两种情况:
\[
m(i,j)=
\begin{cases}
\max\bigl\{m(i+1,j),\ m(i+1,j-w_i)+v_i\bigr\}, & j\ge w_i,\\[6pt]
m(i+1,j), & 0\le j < w_i.
\end{cases}
\]

\textbf{解释:}
\begin{itemize}
  \item 若 $j<w_i$,容量不足,物品 $i$ 不能选,只能继承后续子问题:$m(i,j)=m(i+1,j)$;
  \item 若 $j\ge w_i$,可选择:
  不选 $i$ 得 $m(i+1,j)$;选 $i$ 得 $m(i+1,j-w_i)+v_i$;
  取两者最大即为最优。
\end{itemize}

\noindent\textbf{算法代码:}

\begin{lstlisting}[language=C,style=codeStyle]
template<class Type>
void knapsack(Type v[], int w[], int c, int n, Type **m)
{
    int jMax = min(w[n] - 1, c);
    // 初始化第 n 个物品这一行:m[n][j]
    for (int j = 0; j <= jMax; j++)
        m[n][j] = 0;
    for (int j = w[n]; j <= c; j++)
        m[n][j] = v[n];
    // 从 i=n-1 递推到 2(最后单独处理 i=1)
    for (int i = n - 1; i > 1; i--) {
        jMax = min(w[i] - 1, c);
        // j < w[i] 时装不下物品 i,只能继承 m[i+1][j]
        for (int j = 0; j <= jMax; j++)
            m[i][j] = m[i+1][j];
        // j >= w[i] 时,取“选/不选”两种情况的最大值
        for (int j = w[i]; j <= c; j++)
            m[i][j] = max(m[i+1][j], m[i+1][j - w[i]] + v[i]);
    }
    // 单独处理 i=1(只计算 m[1][c])
    m[1][c] = m[2][c];
    if (c >= w[1])
        m[1][c] = max(m[1][c], m[2][c - w[1]] + v[1]);
}
\end{lstlisting}

\textbf{复杂度分析:}
需要填充大部分 $m[i][j]$(约 $n\times c$ 个状态),
每个状态 $O(1)$ 转移,
因此时间复杂度为
$O(nc),$
空间复杂度为二维表
$O(nc).$

\subsection{最优二叉搜索树(Optimal Binary Search Tree)}

\textbf{问题描述:}
给定有序关键字 $k_1<k_2<\cdots<k_n$。
查找过程中可能命中关键字,也可能命中相邻关键字之间的“失败区间”。
设成功查找 $k_i$ 的概率(或权值)为 $a_i$,
失败查找落在区间之间的概率(或权值)为 $b_i$(通常 $b_0,\dots,b_n$)。
要求构造一棵二叉搜索树,使得\textbf{平均查找代价}(期望比较次数)最小。

\begin{lstlisting}[language=C,style=codeStyle]
void OptimalBinarySearchTree(int a[], int b[], int n, int **m, int **s, int **w)
{
    // 初始化空区间:m[i+1][i] = 0,并给 w 的初值
    for (int i = 0; i <= n; i++) {
        w[i+1][i] = a[i];   // 以该课程版本的定义为准
        m[i+1][i] = 0;
    }
    // r 表示区间长度:j = i + r
    for (int r = 0; r < n; r++) {
        for (int i = 1; i <= n - r; i++) {
            int j = i + r;
            // 计算 w[i][j]
            w[i][j] = w[i][j-1] + a[j] + b[j];
            // 先令 k=i 作为初值
            m[i][j] = m[i+1][j];
            s[i][j] = i;
            // 枚举 k=i+1..j,寻找最优根
            for (int k = i+1; k <= j; k++) {
                int t = m[i][k-1] + m[k+1][j];
                if (t < m[i][j]) {
                    m[i][j] = t;
                    s[i][j] = k;
                }
            }
            // 最后统一加上 w[i][j]
            m[i][j] += w[i][j];
        }
    }
}
\end{lstlisting}

\textbf{复杂度分析:}
共有 $O(n^2)$ 个区间状态 $(i,j)$,
每个状态枚举 $k$ 需要 $O(n)$,
因此时间复杂度为$O(n^3),$空间复杂度为三张 $n\times n$ 表,为$O(n^2).$

\subsection{最大子段和(Maximum Subarray Sum)}

\textbf{问题描述:}
给定长度为 $n$ 的整数序列 $a[1..n]$,求其\textbf{连续子段}的最大和,即
$\max_{1\le l\le r\le n}\ \sum_{k=l}^{r} a[k].$

\textbf{状态定义:}
令$b[j]$
表示\textbf{以 $a[j]$ 结尾}的最大子段和,则有递推:
$$b[j]=\max\{b[j-1]+a[j],\ a[j]\},\qquad 1\le j\le n.$$

最终答案为$\max_{1\le j\le n} b[j].$

\begin{lstlisting}[language=C,style=codeStyle]
int MaxSum(int n, int *a)
{
    int sum = 0;   // 记录目前全局最大
    int b = 0;     // 记录以当前位置结尾的最大子段和
    for (int i = 1; i <= n; i++) {
        if (b > 0) b = b + a[i];
        else       b = a[i];
        if (sum < b) sum = b;
    }
    return sum;
}
\end{lstlisting}

\textbf{复杂度:}
时间复杂度 $O(n),$ 空间复杂度 $O(1).$

\subsection{最大子矩阵和(Maximum Submatrix Sum)}

\textbf{问题描述:}
给定一个 $m\times n$ 的整数矩阵 $A$,
求其元素和最大的子矩阵(连续行+连续列)。

\textbf{核心思想:}
固定子矩阵的上边界行 $i$ 与下边界行 $j$($i\le j$),
把第 $i..j$ 行之间的列元素累加成一个长度为 $n$ 的数组
$b[k]=\sum_{t=i}^{j} A[t][k].$则此时“最佳子矩阵”在列方向上等价于求数组 $b[1..n]$ 的最大子段和。
对所有 $(i,j)$ 枚举即可求最优。

\begin{lstlisting}[language=C,style=codeStyle]
int MaxSum2(int m, int n, int **a)
{
    int sum = 0;
    int *b = new int[n+1];
    for (int i = 1; i <= m; i++) {
        for (int k = 1; k <= n; k++)
            b[k] = 0;
        for (int j = i; j <= m; j++) {
            for (int k = 1; k <= n; k++)
                b[k] += a[j][k];
            int max = MaxSum(n, b);
            if (max > sum) sum = max;
        }
    }
    return sum;
}
\end{lstlisting}

\textbf{复杂度:}
共有 $O(m^2)$ 组行边界 $(i,j)$,
每次更新 $b$ 需要 $O(n)$,求最大子段和也需要 $O(n)$,
所以时间复杂度为
$O(m^2 n),$
额外空间为数组 $b$,即 $O(n).$

\subsection{最大 $m$ 子段和(Maximum $m$ Disjoint Subarrays Sum)}

\textbf{问题描述:}
给定序列 $a[1..n]$,要求选取\textbf{恰好 $m$ 段互不重叠的连续子段},
使得这些子段元素和的总和最大。

\begin{lstlisting}[language=C,style=codeStyle]
int MaxSum(int m, int n, int *a)
{
    if (n < m || m < 1) return 0;
    int *b = new int[n+1];
    int *c = new int[n+1];
    b[0] = 0;
    c[1] = 0;
    for (int i = 1; i <= m; i++) {
        b[i] = b[i-1] + a[i];
        c[i-1] = b[i];
        int maxv = b[i];
        for (int j = i+1; j <= n-m+i; j++) {
            b[j] = (b[j-1] > c[j-1] ? b[j-1] + a[j] : c[j-1] + a[j]);
            c[j-1] = maxv;
            if (maxv < b[j]) maxv = b[j];
        }
        c[i+n-m] = maxv;
    }
    int sum = 0;
    for (int j = m; j <= n; j++)
    if (sum < b[j]) sum = b[j];
    return sum;
}
\end{lstlisting}

\textbf{复杂度:}
两重循环规模约为 $m\times n$,
因此时间复杂度为 $O(mn),$
额外空间为 $b,c$ 两个数组,为 $O(n).$



