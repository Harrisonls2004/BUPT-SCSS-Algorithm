\section{时间复杂度与渐近符号}

\subsection{时间复杂度的基本概念}

设算法输入规模为 $n$,输入实例为 $I$,算法在输入 $I$ 上的运行时间记为 $T(I)$。

\textbf{1.最坏情况时间复杂度:}
$
T_{\max}(n) = \max_{|I|=n} T(I).
$

\textbf{解释:} 在所有规模为 $n$ 的输入中,选取运行时间最长的那个输入,其运行时间作为算法在规模 $n$ 下的时间复杂度。

\textbf{意义:} 最坏情况时间复杂度给出了算法性能的上界,是算法分析中最常用、最保守、也是最安全的度量方式。

\textbf{2.最好情况时间复杂度:}
$
T_{\min}(n) = \min_{|I|=n} T(I).
$

\textbf{解释:} 在所有规模为 $n$ 的输入中,选取运行时间最短的那个输入。

\textbf{说明:} 最好情况通常过于理想,不能反映算法的真实性能,因此在理论分析中参考价值较小。

\textbf{3.平均情况时间复杂度:}
$
T_{\text{avg}}(n) = \sum_{|I|=n} P(I)\,T(I).
$

其中 $P(I)$ 表示输入 $I$ 出现的概率,且满足
$
\sum_{|I|=n} P(I) = 1.
$

\textbf{解释:} 平均情况时间复杂度是对所有输入运行时间的加权平均。

\textbf{说明:} 平均情况分析通常需要对输入分布作概率假设,分析过程复杂,因此在实际算法分析中使用较少。

\subsection{渐近符号的定义}

渐近符号用于刻画当 $n \to \infty$ 时,函数增长速度的数量级,忽略常数因子和低阶项。

\textbf{1.大 $O$ 记号(渐近上界):} $f(n) = O(g(n))$ 当且仅当存在正常数 $c > 0$ 和自然数 $n_0$,使得
$$
\forall n \ge n_0,\quad 0 \le f(n) \le c\,g(n).
$$

\textbf{解释:} 当 $n$ 足够大时,函数 $f(n)$ 的增长速度不会超过 $g(n)$ 的某个常数倍。

\textbf{含义:} 大 $O$ 记号给出了函数增长速度的上界。

\textbf{2.$\Omega$ 记号(渐近下界):} $f(n) = \Omega(g(n))$ 当且仅当存在正常数 $c > 0$ 和自然数 $n_0$,使得
$$
\forall n \ge n_0,\quad f(n) \ge c\,g(n).
$$

\textbf{解释:} 当 $n$ 足够大时,函数 $f(n)$ 的增长速度至少不小于 $g(n)$ 的某个常数倍。

\textbf{含义:} $\Omega$ 记号给出了函数增长速度的下界。

\textbf{3.$\Theta$ 记号(紧确渐近界):} $f(n) = \Theta(g(n))$ 当且仅当 $f(n)=O(g(n))$ 且 $f(n)=\Omega(g(n))$。等价地,存在正常数 $c_1, c_2 > 0$ 和自然数 $n_0$,使得
$$
\forall n \ge n_0,\quad c_1 g(n) \le f(n) \le c_2 g(n).
$$

\textbf{解释:} 函数 $f(n)$ 与 $g(n)$ 具有相同的渐近增长阶,二者在数量级上是等价的。

\textbf{4.小 $o$ 记号(非紧上界):} $f(n) = o(g(n))$ 当且仅当
$$
\lim_{n \to \infty} \frac{f(n)}{g(n)} = 0.
$$

\textbf{解释:} $f(n)$ 的增长速度严格慢于 $g(n)$,即 $f(n)$ 相对于 $g(n)$ 可以忽略。

\textbf{5.小 $\omega$ 记号(非紧下界):} $f(n) = \omega(g(n))$ 当且仅当
$$
\lim_{n \to \infty} \frac{f(n)}{g(n)} = +\infty.
$$

\textbf{解释:} $f(n)$ 的增长速度严格快于 $g(n)$。

\subsection{渐近符号的运算性质}

\textbf{1.加法法则:}
$
O(f(n)) + O(g(n)) = O(\max\{f(n), g(n)\}).
$

\textbf{解释:} 多个子过程顺序执行时,总时间复杂度由增长速度最快的那一项决定。

\textbf{2.乘法法则:}
$
O(f(n)) \cdot O(g(n)) = O(f(n)g(n)).
$

\textbf{解释:} 当一个过程嵌套在另一个过程中执行时,时间复杂度等于二者复杂度的乘积。

\textbf{3.常见等价关系:}
$
O(f(n)) = O(c f(n)), c > 0.
$

$
\log_a n = \Theta(\log_b n), a,b>1.
$

\textbf{说明:} 渐近分析中忽略常数因子与对数底数的差异。

\subsection{上述定理证明}

\textbf{1.证明:$O(f)+O(g)=O(f+g)$}

设 $F(n)=O(f(n))$,则存在自然数 $n_1$ 与正常数 $c_1>0$,当 $n\ge n_1$ 时有
$$
F(n)\le c_1 f(n).
$$

同理,若 $G(n)=O(g(n))$,则存在自然数 $n_2$ 与正常数 $c_2>0$,当 $n\ge n_2$ 时有
$$
G(n)\le c_2 g(n).
$$

当 $n\ge n_0:=\max\{n_1,n_2\}$ 时,两式同时成立,因此
$$
F(n)+G(n)\le c_1 f(n)+c_2 g(n).
$$

令
$$
c_3=\max\{c_1,c_2\},
$$

则有
$$
c_1 f(n)+c_2 g(n)\le c_3 f(n)+c_3 g(n)\le c_3\bigl(f(n)+g(n)\bigr).
$$

于是当 $n\ge n_0$ 时
$$
F(n)+G(n)\le c_3\bigl(f(n)+g(n)\bigr),
$$

从而当 $n\ge n_0$ 时
$$
F(n)+G(n)\le c_3\bigl(f(n)+g(n)\bigr),
$$

因此
$$
O(f(n))+O(g(n))=O\bigl(f(n)+g(n)\bigr).
$$

\textbf{2.由 $O(f(n))+O(g(n))=O(f(n)+g(n))$ 推出 $O(f(n))+O(g(n))=O(\max\{f(n),g(n)\})$}

已证
$$
O(f(n))+O(g(n))=O(f(n)+g(n)).
$$

因此存在正常数 $C_3>0$ 与自然数 $n_0$,使得当 $n\ge n_0$ 时
$$
O(f(n))+O(g(n))\le C_3\bigl(f(n)+g(n)\bigr).
$$

又因为对任意 $n$(默认 $f(n),g(n)\ge 0$)都有
$$
f(n)+g(n)\le 2\max\{f(n),g(n)\},
$$

代入上式得当 $n\ge n_0$ 时
$$
O(f(n))+O(g(n))
\le 2C_3\max\{f(n),g(n)\}.
$$

由于大 $O$ 记号忽略正常数倍,故
$$
O(f(n))+O(g(n)) = O(\max\{f(n),g(n)\}).
$$

\textbf{3.证明:$O(f(n))\cdot O(g(n))=O(f(n)g(n))$}

令 $f_1(n)=O(f(n))$,则存在自然数 $n_1$ 与正常数 $c_1>0$,当 $n\ge n_1$ 时有
$$
f_1(n)\le c_1 f(n).
$$

同理,若 $g_1(n)=O(g(n))$,则存在自然数 $n_2$ 与正常数 $c_2>0$,当 $n\ge n_2$ 时有
$$
g_1(n)\le c_2 g(n).
$$

当 $n\ge n_0:=\max\{n_1,n_2\}$ 时,两式同时成立,相乘得到
$$
f_1(n)\,g_1(n)\le (c_1 f(n))(c_2 g(n)) = c_3 f(n)g(n),
$$

其中
$$
c_3=c_1c_2.
$$

因此当 $n\ge n_0$ 时
$$
f_1(n)\,g_1(n)\le c_3 f(n)g(n),
$$

从而
$$
O(f(n))\cdot O(g(n)) = O(f(n)g(n)).
$$
