\section{递归与分治}

\subsection{阶乘函数(Factorial)}

\textbf{定义与递归式:}
阶乘用于表示从 $1$ 到 $n$ 的连乘积(约定 $0!=1$):  
$n! = \prod_{k=1}^{n} k$。  
递归定义通常写成:
$$
n! =
\begin{cases}
1, & n=0,\\
n\cdot (n-1)!, & n>0.
\end{cases}
$$
$$
n! = 1\cdot 2\cdot \cdots \cdot (n-1)\cdot n
= n\cdot \bigl(1\cdot 2\cdot \cdots \cdot (n-1)\bigr)
= n\cdot (n-1)!.
$$

\textbf{递归计算的时间复杂度:}若用递归实现,每次调用把 $n$ 减 $1$:  
$T(n)=T(n-1)+O(1)$,展开得 $T(n)=O(n)$。递归深度为 $n$,因此额外栈空间也为 $O(n)$。

\textbf{等价的迭代写法:}循环从 $1$ 乘到 $n$,时间仍为 $O(n)$,空间可做到 $O(1)$。

\subsection{斐波那契数列}
$$
F(n)=
\begin{cases}
1, & n=0,\\
1, & n=1,\\
F(n-1)+F(n-2), & n>1.
\end{cases}
$$

\textbf{代码:}$\texttt{if (n<=1) return 1; else return F(n-1)+F(n-2);}$  

\textbf{朴素递归时间复杂度:}设递归实现的运行时间为 $T(n)$,则$T(n)=T(n-1)+T(n-2)+O(1).$
它与斐波那契增长同阶,解的数量级为指数级:
$
T(n)=\Theta(\varphi^{\,n}),
\qquad
\varphi=\frac{1+\sqrt{5}}{2}.
$
因此朴素递归在 $n$ 稍大时就会爆炸性变慢。

\subsection{Ackermann函数(双递归函数)}

$$
A(1,0)=2,
$$
$$
A(0,m)=1,\quad m\ge 0,
$$
$$
A(n,0)=n+2,\quad n\ge 2,
$$
$$
A(n,m)=A\bigl(A(n-1,m),\,m-1\bigr),\quad m\ge 1.
$$

\subsection{排列问题}

\begin{lstlisting}[language=C,style=codeStyle]
void Perm(Type list[], int k, int m) {
    if (k == m) {                      // 递归终止条件:只剩一个元素
        for (int i = 0; i <= m; i++)   // 输出当前排列
            cout << list[i];
        cout << endl;
    } 
    else {                             // 递归生成排列
        for (int i = k; i <= m; i++) {
            Swap(list[k], list[i]);    // 交换元素到当前位置k
            Perm(list, k + 1, m);      // 递归生成剩余元素的排列
            Swap(list[k], list[i]);    // 恢复原始顺序(回溯)
        }
    }
}
inline void Swap(Type &a, Type &b) {
    Type temp = a;
    a = b;
    b = temp;
}
\end{lstlisting}

\textbf{时间复杂度:}
全排列的总数为 $n!$,  
任何生成全部排列的算法,其时间复杂度至少为
$\Omega(n!).$

\subsection{整数划分问题}

\textbf{问题定义:}把正整数 $n$ 表示成若干个正整数之和,且加数\textbf{不考虑顺序},称为 $n$ 的一个整数划分。

\begin{lstlisting}[language=C,style=codeStyle]
int q(int n, int m) {
    if ((n < 1) || (m < 1)) return 0;     // 非法输入返回0
    if ((n == 1) || (m == 1)) return 1;  // 基准情况
    if (n < m) return q(n, n);            // 划分数不能超过n本身
    if (n == m) return 1 + q(n, n-1);
    if (n > m) return q(n, m-1) + q(n-m, m);  
\end{lstlisting}

\subsection{求解线性递推关系}

\textbf{问题描述:}

给定递推关系
$$
\begin{cases}
X_{n+1}=X_n+12X_{n-1}, & n\ge 1,\\
X_0=1,\quad X_1=0.5.
\end{cases}
$$

\textbf{判断递推类型}
该递推关系只包含 $X_{n+1},X_n,X_{n-1}$,系数为常数,且右端无常数项,因此是\textbf{二阶常系数齐次线性递推关系}。

\textbf{(1)写出特征方程}

设 $X_n=r^n$,代入得
$
r^{n+1}=r^n+12r^{n-1}.
$

两边除以 $r^{n-1}$ 得$r^2=r+12,$
即$r^2-r-12=0.$

\textbf{(2)求解特征方程}

$(r-4)(r+3)=0,$
$r_1=4,r_2=-3.$

\textbf{(3)通解形式}

$X_n=C_1(-3)^n+C_2 4^n.$

\textbf{(4)利用初始条件求常数}
$
\begin{cases}
C_1+C_2=1,\\
-3C_1+4C_2=0.5.
\end{cases}
$

\textbf{(5)解方程组}$C_1=0.5,C_2=0.5.$

\textbf{(6)最终通项公式}

$
\boxed{X_n=0.5(-3)^n+0.5\cdot 4^n}
$

\textbf{理解要点总结:}
假设指数解;构造特征方程;不同特征根对应指数项线性组合;初始条件用于唯一确定常数。

\subsection{汉诺塔问题(Tower of Hanoi)}

\textbf{问题描述:}将 $n$ 个大小不同的圆盘从起始柱 $a$ 移动到目标柱 $b$,在移动过程中可以借助辅助柱 $c$,并且必须满足以下规则:每次只能移动一个圆盘;任意时刻,大圆盘不能放在小圆盘上面。

\begin{lstlisting}[language=C,style=codeStyle]
void hanoi(int n, int a, int b, int c) {
    if (n > 0) {
        hanoi(n - 1, a, c, b);  // 将n-1个盘子从a移到c(借助b)
        move(a, b);             // 将第n个盘子从a直接移到b
        hanoi(n - 1, c, b, a);  // 将n-1个盘子从c移到b(借助a)
    }
}
\end{lstlisting}

\textbf{时间复杂度分析:}设移动 $n$ 个圆盘所需的时间为 $T(n)$,则有递推关系
$T(n)=2T(n-1)+1.$解得:$T(n)=2^n-1$.因此,汉诺塔问题的时间复杂度为$O(2^n)$

\textbf{空间复杂度分析:}递归调用的最大深度为 $n$,因此递归栈所需的空间复杂度为$O(n).$

\subsection{分治算法时间复杂度}
$$
T(n)=
\begin{cases}
O(1), & n=1,\\
k\,T\!\left(\dfrac{n}{m}\right)+f(n), & n>1.
\end{cases}
$$
其中:$k$子问题个数;$n/m$每个子问题的规模;$f(n)$当前层的额外计算(分割或合并代价)。

\textbf{重要结论}
\begin{enumerate}
  \item 递归深度为:$\log_m n.$
  \item 叶子结点(最底层子问题)总数量为:$k^{\log_m n}=n^{\log_m k}.$
  \item 分治算法的总时间复杂度可以表示为:
  $$
  T(n)
  =
  n^{\log_m k}
  +
  \sum_{i=0}^{\log_m n -1}
  k^i f\!\left(\frac{n}{m^i}\right).
  $$
\end{enumerate}

\textbf{记忆要点:}看\textbf{子问题个数}$k$;看\textbf{规模缩小倍数}$m$;先算 $n^{\log_m k}$(叶子结点代价);再看 $f(n)$ 在各层累加后的大小;谁占主导,谁就是最终时间复杂度。

\subsection{二分搜索(Binary Search)}
\begin{lstlisting}[language=C,style=codeStyle]
int BinarySearch(Type a[], const Type &x, int l, int r)
{
    while (l <= r) {
        int m = (l + r) / 2;   // 计算中点下标
        if (x == a[m]) 
            return m;         // 找到目标,返回位置
        if (x < a[m]) 
            r = m - 1;        // 目标在左半区,缩小右边界
        else 
            l = m + 1;        // 目标在右半区,缩小左边界
    }
    return -1;                // 查找失败
}
\end{lstlisting}

\textbf{时间复杂度:}
设数组长度为 $n$,每次比较后查找区间规模减半:$n \rightarrow \frac{n}{2} \rightarrow \frac{n}{4} \rightarrow \cdots \rightarrow 1.$
因此查找次数为$\log_2 n,$
二分搜索的时间复杂度为$O(\log n).$

\textbf{空间复杂度:}该实现为迭代版本,只使用常数个辅助变量,因此空间复杂度为$O(1).$

\subsection{大整数乘法(Karatsuba 算法)}

\textbf{问题背景:}设 $X,Y$ 为两个 $n$ 位的大整数,若直接相乘,时间复杂度为$O(n^2).$

\textbf{分治思想:}
将两个 $n$ 位整数拆分为高位和低位(假设 $n$ 为偶数):$X = a\cdot 2^{n/2} + b, Y = c\cdot 2^{n/2} + d.$

直接展开乘积:$XY = ac\cdot 2^n + (ad+bc)\cdot 2^{n/2} + bd.$

\textbf{关键改进(Karatsuba 思想):}注意到$ad+bc = (a+b)(c+d) - ac - bd.$因此只需计算以下三个乘法:$ac$、$bd$、$(a+b)(c+d)$其余部分只需要加减运算。

\textbf{递归时间复杂度:}

Karatsuba 算法的递归式为:
$$
T(n)=
\begin{cases}
O(1), & n=1,\\
3T(n/2)+O(n), & n>1.
\end{cases}
$$

由主定理可得:$T(n)=O\!\left(n^{\log_2 3}\right)\approx O(n^{1.585}),
$优于普通的大整数乘法 $O(n^2)$。

\subsection{Strassen 矩阵乘法}

\textbf{问题背景:}设 $A,B$ 为 $n\times n$ 矩阵,普通矩阵乘法的时间复杂度为:$O(n^3).$

\textbf{分块表示:}

将矩阵划分为四个子块
$$
A=
\begin{pmatrix}
A_{11} & A_{12}\\
A_{21} & A_{22}
\end{pmatrix},
\qquad
B=
\begin{pmatrix}
B_{11} & B_{12}\\
B_{21} & B_{22}
\end{pmatrix}.
$$

\textbf{Strassen 的 7 个中间矩阵}

\[
\begin{aligned}
M_1 &= A_{11}(B_{12}-B_{22}),\\
M_2 &= (A_{11}+A_{12})B_{22},\\
M_3 &= (A_{21}+A_{22})B_{11},\\
M_4 &= A_{22}(B_{21}-B_{11}),\\
M_5 &= (A_{11}+A_{22})(B_{11}+B_{22}),\\
M_6 &= (A_{12}-A_{22})(B_{21}+B_{22}),\\
M_7 &= (A_{11}-A_{21})(B_{11}+B_{12}).
\end{aligned}
\]

\textbf{结果矩阵的计算}

\[
\begin{aligned}
C_{11} &= M_5 + M_4 - M_2 + M_6,\\
C_{12} &= M_1 + M_2,\\
C_{21} &= M_3 + M_4,\\
C_{22} &= M_5 + M_1 - M_3 - M_7.
\end{aligned}
\]

\textbf{时间复杂度:}Strassen 矩阵乘法的递归式为$T(n)=7T(n/2)+O(n^2).$
由主定理可得:
$T(n)=O\!\left(n^{\log_2 7}\right)\approx O(n^{2.81}),
$优于普通矩阵乘法的 $O(n^3)$。

\subsection{棋盘覆盖算法}
要求:对一个 $2^k\times 2^k$ 的棋盘进行覆盖,
其中恰好有一个特殊方格(缺口),其余方格需用 L 型骨牌覆盖。算法采用分治与递归思想,始终保持:
\textbf{每一个递归子棋盘中恰好只有一个缺口}。

\begin{lstlisting}[language=C,style=codeStyle]
#define N 8
int board[N][N];
int tile = 1; // 骨牌编号

// 参数说明:
// tr, tc:当前子棋左上角位置
// dr, dc:当前特殊方格坐标(绝对坐标)
// size:当前子棋盘规模(边长)
void chessBoard(int tr, int tc, int dr, int dc, int size) {
    if (size == 1) return;
    int t = tile++;
    int s = size / 2;
    
    // 左上子棋盘
    if (dr < tr + s && dc < tc + s) {
        chessBoard(tr, tc, dr, dc, s);
    } else {
        board[tr + s - 1][tc + s - 1] = t;
        chessBoard(tr, tc, tr + s - 1, tc + s - 1, s);
    }
    
    // 右上子棋盘
    if (dr < tr + s && dc >= tc + s) {
        chessBoard(tr, tc + s, dr, dc, s);
    } else {
        board[tr + s - 1][tc + s] = t;
        chessBoard(tr, tc + s, tr + s - 1, tc + s, s);
    }
    
    // 左下子棋盘
    if (dr >= tr + s && dc < tc + s) {
        chessBoard(tr + s, tc, dr, dc, s);
    } else {
        board[tr + s][tc + s - 1] = t;
        chessBoard(tr + s, tc, tr + s, tc + s - 1, s);
    }
    
    // 右下子棋盘
    if (dr >= tr + s && dc >= tc + s) {
        chessBoard(tr + s, tc + s, dr, dc, s);
    } else {
        board[tr + s][tc + s] = t;
        chessBoard(tr + s, tc + s, tr + s, tc + s, s);
    }
}
\end{lstlisting}

\subsection{合并排序(Merge Sort)}
合并排序是一种典型的分治算法。其基本思想是:\textbf{将待排序序列不断划分为两个子序列,
分别排序后,再将两个有序子序列合并为一个有序序列}。

\textbf{1.递归排序函数}
\begin{lstlisting}[language=C,style=codeStyle]
void mergeSort(Type a[], int left, int right)
{
    if (left < right) {
        int i = (left + right) / 2;
        mergeSort(a, left, i);
        mergeSort(a, i + 1, right);
        merge(a, b, left, i, right);
        copy(a, b, left, right);
    }
}
\end{lstlisting}

\textbf{2.合并函数}

\begin{lstlisting}[language=C,style=codeStyle]
void merge(Type a[], Type d[], int l, int m, int r)
{
    int i = l, j = m + 1, k = l;
    while (i <= m && j <= r) {
        if (a[i] <= a[j]) d[k++] = a[i++];
        else d[k++] = a[j++];
    }

    if (i > m)
        for (int q = j; q <= r; q++)  d[k++] = a[q];
    else
        for (int q = i; q <= m; q++)  d[k++] = a[q];
}
\end{lstlisting}

\textbf{时间复杂度:}合并排序的递归关系为$T(n) = 2T(n/2) + O(n).$由主定理可得$T(n) = O(n \log n).$

\textbf{空间复杂度:}由于合并过程需要额外的辅助数组,合并排序的空间复杂度为$O(n).$

\subsection{快速排序(Quick Sort)}

快速排序是一种典型的分治排序算法,其基本思想是:
\textbf{选取一个基准元素,将序列划分为左右两部分,
使得左边元素不大于基准,右边元素不小于基准,
然后分别对左右两部分递归排序。}

\textbf{1.快速排序的递归框架}

\begin{lstlisting}[language=C,style=codeStyle]
void QuickSort(Type a[], int p, int r)
{
    if (p < r) {
        int q = Partition(a, p, r);
        QuickSort(a, p, q - 1);
        QuickSort(a, q + 1, r);
    }
}
\end{lstlisting}

\textbf{2.划分函数(Partition)}

\begin{lstlisting}[language=C,style=codeStyle]
int Partition(Type a[], int p, int r)
{
    int i = p, j = r + 1;
    Type x = a[p];
    while (true) {
        while (a[++i] < x);
        while (a[--j] > x);
        if (i >= j) break;
        Swap(a[i], a[j]);
    }
    a[p] = a[j];
    a[j] = x;
    return j;
}
\end{lstlisting}

\textbf{3.随机化划分}

\begin{lstlisting}[language=C,style=codeStyle]
int RandomizedPartition(Type a[], int p, int r)
{
    int i = Random(p, r);
    Swap(a[i], a[p]);
    return Partition(a, p, r);
}
\end{lstlisting}

\textbf{时间复杂度:}
在平均情况下,快速排序每次划分都能较均匀地分割数组,
递归深度为 $O(\log n)$,
每一层划分操作的代价为 $O(n)$,
因此平均时间复杂度为:$O(n \log n).$
在最坏情况下(例如数组已经有序且基准选择不当),每次只能划分出一个规模为 $n-1$ 的子问题,时间复杂度退化为:$O(n^2).$

\textbf{空间复杂度:}快速排序主要消耗递归栈空间,平均情况下递归深度为 $O(\log n)$,因此空间复杂度为:$O(\log n).$

\subsection{随机选择算法(Randomized Select)}

随机选择算法用于在数组中查找第 $k$ 小元素,其思想来源于快速排序中的划分操作。

\begin{lstlisting}[language=C,style=codeStyle]
RSelect(a, p, r, k)
{
    if (p == r) return a[p];
    mid = RandomizedPartition(a, p, r);
    if (mid == k) return a[mid];
    else if (mid > k)
        return RSelect(a, p, mid - 1, k);
    else
        return RSelect(a, mid + 1, r, k);
}
\end{lstlisting}

由于每一轮递归只进入一个子区间,该算法的平均时间复杂度为:$O(n).$

\subsection{寻找第 $k$ 小元素(Randomized-Select 算法)}

\textbf{随机划分函数 RandomizedPartition}

\begin{lstlisting}[language=C,style=codeStyle]
template<class Type>
int RandomizedPartition(Type a[], int p, int r) {
    int i = Random(p, r);
    Swap(a[i], a[p]);
    return Partition(a, p, r);
}
\end{lstlisting}

\textbf{划分函数 Partition}

\begin{lstlisting}[language=C,style=codeStyle]
template<class Type>
int Partition(Type a[], int p, int r)
{
    int i = p, j = r + 1;
    Type x = a[p];
    while (true) {
        while (a[++i] < x && i < r);
        while (a[--j] > x);
        if (i >= j) break;
        Swap(a[i], a[j]);
    }
    a[p] = a[j];
    a[j] = x;
    return j;
}
\end{lstlisting}

\textbf{随机化选择函数 RandomizedSelect}

\begin{lstlisting}[language=C,style=codeStyle]
template<class Type>
Type RandomizedSelect(Type a[], int p, int r, int k) {
    if (p == r) return a[p];
    int i = RandomizedPartition(a, p, r);
    int j = i - p + 1;
    if (k <= j)
        return RandomizedSelect(a, p, i, k);
    else
        return RandomizedSelect(a, i + 1, r, k - j);
}
\end{lstlisting}

\textbf{算法复杂度分析:}在平均情况下,每一次划分都能较均匀地缩小问题规模,因此随机化选择算法的期望时间复杂度为:$O(n)$.相比完全排序所需的 $O(n \log n)$ 时间复杂度,该算法在仅需寻找第 $k$ 小元素时具有更高的效率。

\subsection{循环赛问题}

循环赛(Round-Robin Tournament)问题的目标是:给定 $n$ 个选手(或队伍),安排比赛日程,使得每一对选手恰好比赛一次,并且每天(每一轮)每位选手只进行一场比赛。  
在经典模型中,若 $n=2^k$,可以构造一个 $n \times n$ 的表格 $a$ 来描述日程安排,其中 $a[i][j]$ 表示第 $i$ 个选手在第 $j$ 轮的对手编号(不同教材可能从第 1 轮到第 $n-1$ 轮,这里代码按表格整体递推生成)。

\begin{lstlisting}[language=C,style=codeStyle]
void Table(int k, int **a)
{
    int n = 1;
    for (int i = 1; i <= k; i++)
        n *= 2;                 // 计算 n = 2^k
    for (int i = 1; i <= n; i++)
        a[1][i] = i;            // 初始化第一行(作为递推基底)
    int m = 1;                  // 当前已构造子表规模为 m
    for (int s = 1; s <= k; s++) {
        n /= 2;                 // 每次递推对应划分成若干块
        for (int t = 1; t <= n; t++) {
            for (int i = m + 1; i <= 2 * m; i++) {
                for (int j = m + 1; j <= 2 * m; j++) {
                    a[i][j + (t - 1) * m * 2 - m] =
                        a[i - m][j + (t - 1) * m * 2 - 2 * m];
                }
            }
        }
        m *= 2;                // 子表规模翻倍:m -> 2m
    }
}
\end{lstlisting}
\textbf{复杂度分析:} 该算法需要填充 $n \times n$ 的表格单元,每个单元最多被常数次赋值,因此时间复杂度为:$O(n^2)$;空间上需要保存整个赛程表,同样为:$O(n^2)$.