\section{贪心算法}

贪心算法在求解优化问题时,每一步都作出一个当前看来“最优”的选择,
期望通过一系列局部最优决策得到全局最优解。
贪心算法的关键在于:\textbf{问题必须具有贪心选择性质和最优子结构}。

\subsection{活动选择问题}

\textbf{问题描述:}
设有 $n$ 个活动 $A_1,A_2,\dots,A_n$,
每个活动 $A_i$ 有开始时间 $s_i$ 和结束时间 $f_i$,
同一时间只能进行一个活动。
目标是从中选择尽可能多的\textbf{互不冲突的活动}。

\textbf{贪心策略:}按活动的\textbf{结束时间从小到大排序};每次选择\textbf{当前结束时间最早且与已选活动不冲突的活动}。该策略可以保证给后续活动留下尽可能多的时间,因此是全局最优的。

\begin{lstlisting}[language=C,style=codeStyle]
void GreedySelector(int n, int s[], int f[], bool A[])
{
    A[1] = true;          // 选择第一个活动
    int j = 1;
    for (int i = 2; i <= n; i++) {
        if (s[i] >= f[j]) {
            A[i] = true; // 选择活动 i
            j = i;
        } else {
            A[i] = false;
        }
    }
}
\end{lstlisting}

\textbf{时间复杂度:}
若活动已排序,则算法时间复杂度为$O(n).$若需要先排序,则总体复杂度为$O(n\log n).$

\subsection{分数背包问题(Fractional Knapsack)}

\textbf{问题描述:}
给定 $n$ 个物品,第 $i$ 个物品重量为 $w_i$,价值为 $v_i$,
背包容量为 $M$。与 0--1 背包不同,\textbf{每个物品允许取任意比例}。
目标是在不超过容量 $M$ 的前提下,使装入背包的总价值最大。

\textbf{贪心策略:}按物品的\textbf{单位重量价值} $\frac{v_i}{w_i}$ 从大到小排序;
依次尽可能多地装入单位价值最高的物品;
若剩余容量不足以装入整个物品,则装入该物品的一部分并结束。
该策略能保证每一步都优先获得最大“单位收益”,因此可以得到全局最优解。

\begin{lstlisting}[language=C,style=codeStyle]
void Knapsack(int n, float M, float v[], float w[], float x[])
{
    Sort(n, v, w);    // 按 v[i]/w[i] 从大到小排序
    for (int i = 1; i <= n; i++)
    x[i] = 0;
    float c = M;
    int i;
    for (i = 1; i <= n; i++) {
        if (w[i] > c) break;   // 剩余容量不足,无法装入整个物品 i
        x[i] = 1;              // 装入整个物品 i
        c -= w[i];
    }
    if (i <= n) x[i] = c / w[i];   // 装入物品 i 的一部分
}
\end{lstlisting}

\textbf{时间复杂度:}
排序耗时 $O(n\log n)$,装包过程为 $O(n)$,
因此总体时间复杂度为
$O(n\log n).$

\subsection{最优装载问题(Optimal Loading)}

\textbf{问题描述:}
有 $n$ 个货箱,第 $i$ 个货箱的重量为 $w_i$,
现有一艘载重能力为 $C$ 的船。
要求选择若干个货箱装入船中,
使得\textbf{装入货箱的数量尽可能多},且总重量不超过船的载重能力 $C$。

\textbf{贪心策略:}优先装载\textbf{重量较小的货箱};
每次选择当前剩余货箱中重量最小者装入船中;
当剩余载重不足以装下下一个货箱时,算法结束。
该策略保证在相同载重条件下,能够装入尽可能多的货箱,因此是最优的。

\begin{lstlisting}[language=C,style=codeStyle]
void Loading(int x[], int w[], int C, int n)
{
    int *t = new int[n+1];     // t[i] 存放排序后的货箱下标
    Sort(w, t, n);             // 按重量 w 从小到大排序,结果存入 t
    for (int i = 1; i <= n; i++)
        x[i] = 0;              // 初始化:不装任何货箱
    for (int i = 1; i <= n && w[t[i]] <= C; i++) {
        x[t[i]] = 1;           // 装入编号为 t[i] 的货箱
        C -= w[t[i]];          // 更新剩余载重
    }
}
\end{lstlisting}

\textbf{时间复杂度:}
排序过程需要 $O(n\log n)$ 时间,
装载过程最多遍历一次货箱,为 $O(n)$,
因此算法的总体时间复杂度为
$O(n\log n).$

\textbf{空间复杂度:}
额外使用了数组 \verb|t| 和 \verb|x|,空间复杂度为
$O(n).$

\subsection{哈夫曼编码(Huffman Coding)}

\textbf{问题描述:}
给定一组字符及其出现频率(或权值)$\{c_1,c_2,\dots,c_n\}$,
要求为每个字符设计一个二进制编码,使得:
编码满足\textbf{前缀码}性质(任一字符的编码不是另一个字符编码的前缀);
所有字符编码后的\textbf{加权路径长度(WPL)}最小。

其中,加权路径长度定义为
$
\text{WPL}=\sum_{i=1}^{n} w_i \cdot l_i,
$
$w_i$ 为字符 $i$ 的频率(权值),$l_i$ 为其编码长度。

\textbf{贪心策略:}
在当前所有结点中,反复选取\textbf{权值最小的两个结点};
将这两个结点合并为一个新结点,新结点的权值为二者权值之和;
将新结点重新插入集合中,重复上述过程,直到只剩一个结点。
该策略保证每一步的局部最优选择(合并最小权值结点)
最终得到全局最优的哈夫曼树。

\begin{lstlisting}[language=C,style=codeStyle]
void Huffman(int w[], int n)
{
    MinHeap<HuffmanNode> Q;
    Initialize(Q, w, n);   // 将 n 个权值初始化到最小堆中
    HuffmanNode x, y, z;
    for (int i = 1; i < n; i++) {
        Q.deleteMin(x);    // 取出权值最小的结点 x
        Q.deleteMin(y);    // 取出权值次小的结点 y
        z.MakeTree(x, y);  // 合并 x 和 y 为一棵新树
        z.weight = x.weight + y.weight;
        Q.insert(z);       // 将新结点插回最小堆
    }
}
\end{lstlisting}

\textbf{时间复杂度:}
最小堆初始化需要 $O(n)$ 时间,
合并过程中共进行 $n-1$ 次循环,
每次包含两次删除最小值和一次插入操作,
每个堆操作时间为 $O(\log n)$,
因此总时间复杂度为
$O(n\log n).$

\subsection{单源最短路径问题(Dijkstra 算法)}

\textbf{问题描述:}
给定一个带非负权值的有向图(或无向图) $G=(V,E)$,
其中每条边 $(u,v)$ 的权值为 $c(u,v)\ge 0$。
指定一个源点 $v\in V$,
要求计算从源点 $v$ 到图中其余各顶点的\textbf{最短路径长度},
并可同时记录对应的最短路径。

\begin{lstlisting}[language=C,style=codeStyle]
void Dijkstra(int n, int v, Type c[][MAXN], Type dist[], int prev[])
{
    bool S[MAXN];
    for (int i = 1; i <= n; i++) {
        dist[i] = c[v][i];   // 初始化距离
        S[i] = false;        // 初始时所有顶点均未加入 S
        if (dist[i] < maxint) prev[i] = v;
        else prev[i] = -1;
    }
    dist[v] = 0;
    S[v] = true;             // 源点加入 S
    for (int i = 1; i < n; i++) {
        Type temp = maxint;
        int u = v;
        // 在 V-S 中寻找 dist 最小的顶点 u
        for (int j = 1; j <= n; j++) {
            if (!S[j] && dist[j] < temp) {
                u = j;
                temp = dist[j];
            }
        }
        S[u] = true;         // 将 u 加入 S
        // 用 u 松弛其邻接点
        for (int j = 1; j <= n; j++) {
            if (!S[j] && c[u][j] < maxint) {
                Type newdist = dist[u] + c[u][j];
                if (newdist < dist[j]) {
                    dist[j] = newdist;
                    prev[j] = u;
                }
            }
        }
    }
}
\end{lstlisting}

\textbf{时间复杂度:}
在该实现中每一轮需在 $O(n)$ 时间内寻找最小的 $dist$;共进行 $n-1$ 轮。
因此总时间复杂度为$O(n^2).$

\subsection{最小生成树(Minimum Spanning Tree)}

给定一个\textbf{连通无向带权图} $G=(V,E)$,
其中每条边 $(u,v)$ 具有权值 $w(u,v)$。
最小生成树(MST)是指一棵:包含图中所有顶点;边数为 $|V|-1$;总权值之和最小的生成树。

下面介绍两种经典的贪心算法:\textbf{Prim 算法}和 \textbf{Kruskal 算法}。

\subsubsection{Prim 算法}

\textbf{基本思想:}
Prim 算法从某个起始顶点出发,
逐步扩展一棵生成树。
在每一步中,选择一条\textbf{连接当前生成树与外部顶点的最小权值边},
将该边及其对应的顶点加入生成树。

\begin{lstlisting}[language=C,style=codeStyle]
void Prim(int n, Type c[][MAXN])
{
    bool S[MAXN];
    Type lowcost[MAXN];
    int closest[MAXN];
    // 初始化:从顶点 1 开始
    for (int i = 1; i <= n; i++) {
        S[i] = false;
        lowcost[i] = c[1][i];
        closest[i] = 1;
    }
    S[1] = true;
    for (int i = 1; i < n; i++) {
        Type min = maxint;
        int j = 1;
        // 选择距离生成树最近的顶点 j
        for (int k = 2; k <= n; k++) {
            if (!S[k] && lowcost[k] < min) {
                min = lowcost[k];
                j = k;
            }
        }
        S[j] = true;   // 将顶点 j 加入生成树
        // 更新其余顶点到生成树的最小距离
        for (int k = 2; k <= n; k++) {
            if (!S[k] && c[j][k] < lowcost[k]) {
                lowcost[k] = c[j][k];
                closest[k] = j;
            }
        }
    }
}
\end{lstlisting}

\textbf{时间复杂度:}该实现中每次选择最小边需 $O(n)$,共进行 $n-1$ 次,因此时间复杂度为$O(n^2)$。

\subsubsection{Kruskal 算法}

\textbf{基本思想:}
Kruskal 算法从\textbf{边的角度}构造最小生成树:将所有边按权值从小到大排序;
依次选择当前权值最小且\textbf{不会形成回路}的边;
直到选取 $|V|-1$ 条边为止。

\begin{lstlisting}[language=C,style=codeStyle]
void Kruskal(int n, int m, Edge edges[])
{
    MinHeap<Edge> H;
    UnionFind U(n);
    for (int i = 1; i <= m; i++)
        H.insert(edges[i]);   // 将所有边加入最小堆
    int k = 0;   // 已选边数
    while (k < n-1 && !H.empty()) {
        Edge x;
        H.deleteMin(x);       // 取出权值最小的边
        int a = U.Find(x.u);
        int b = U.Find(x.v);
        if (a != b) {         // 不形成回路
            k++;
            U.Union(a, b);    // 合并两个连通分量
        }
    }
}
\end{lstlisting}

\textbf{时间复杂度:}边排序(或最小堆)需要 $O(m\log m)$;并查集操作近似为 $O(1)$;
因此总体时间复杂度为$O(m\log m)$。

\textbf{算法比较:} Prim 算法适合\textbf{稠密图};
Kruskal 算法适合\textbf{稀疏图};
二者均为贪心算法,且都能正确求得最小生成树。

\subsection{多机调度问题(Multi-machine Scheduling)}

\textbf{问题描述:}
设有 $n$ 个相互独立的作业 $J_1,J_2,\dots,J_n$,
以及 $m$ 台相同的机器 $M_1,M_2,\dots,M_m$。
第 $i$ 个作业的处理时间为 $p_i$。
每个作业可以在\textbf{任意一台机器}上加工,但\textbf{同一时刻一台机器只能加工一个作业},
且\textbf{作业一旦开始加工不能中断}。目标是:
\textbf{合理安排作业顺序与分配方案,使所有作业完成的总时间(完工时间,makespan)最小。}
该问题是一个经典的 \textbf{NP-完全问题},
目前不存在多项式时间的精确算法,通常采用贪心策略设计近似算法。

\textbf{贪心策略(最长作业优先,LPT):}
将所有作业按处理时间 $p_i$ \textbf{从大到小排序};
依次将当前作业分配给\textbf{当前负载最小(最早空闲)的机器}。
该策略的直观思想是
\textbf{优先处理耗时最长的作业,避免其被推迟到后期造成整体完成时间过大。}

\textbf{算法步骤说明:}
\begin{itemize}
  \item 当 $n \le m$ 时,可直接将每个作业分配给一台机器,
        总完成时间为 $\max\{p_1,p_2,\dots,p_n\}$;
  \item 当 $n > m$ 时:
    \begin{enumerate}
      \item 按作业处理时间从大到小排序;
      \item 维护每台机器的当前完成时间;
      \item 每次选择当前完成时间最小的机器分配下一个作业。
    \end{enumerate}
\end{itemize}

\textbf{时间复杂度:}作业排序需要 $O(n\log n)$;作业分配过程可在 $O(n\log m)$(使用优先队列)或 $O(nm)$ 时间内完成;因此整体时间复杂度为$O(n\log n).$